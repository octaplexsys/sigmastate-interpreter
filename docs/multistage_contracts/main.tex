%\documentclass[11pt]{article}
\documentclass[runningheads]{llncs}
\def\shownotes{1}
\def\notesinmargins{0}

%\usepackage{fullpage}

\usepackage{mathtools,color,xcolor,hyperref,graphicx,wrapfig,listings,array,xspace}
\usepackage{caption}
\usepackage{subcaption}
\usepackage{tikz}
\usetikzlibrary{arrows}

\usepackage{amsfonts}
% https://tex.stackexchange.com/questions/11719/how-do-i-backcolor-in-verbatim
\usepackage{xcolor}
\usepackage{alltt}
% Compensate for fbox sep:
\newcommand\Hi[2][lightgray]{%
	\hspace*{-\fboxsep}%
	\colorbox{#1}{#2}%
	\hspace*{-\fboxsep}%
}

\ifnum\shownotes=1
\ifnum\notesinmargins=1
\newcommand{\authnote}[2]{\marginpar{\parbox{\marginparwidth}{\tiny %
  \textsf{#1 {\textcolor{blue}{notes: #2}}}}}%
  \textcolor{blue}{\textbf{\dag}}}
\else
\newcommand{\authnote}[2]{
  \textsf{#1 \textcolor{blue}{: #2}}}
\fi
\else
\newcommand{\authnote}[2]{}
\fi

\newcommand{\lnote}[1]{{\authnote{\textcolor{orange}{Leo notes}}{#1}}}
\newcommand{\snote}[1]{{\authnote{\textcolor{yellow}{Scalahub notes}}{#1}}}
\newcommand{\knote}[1]{{\authnote{\textcolor{green}{kushti notes}}{#1}}}
\newcommand{\mnote}[1]{{\authnote{\textcolor{red}{Morphic}}{#1}}}
\newcommand{\dnote}[1]{{\authnote{\textcolor{brown}{Dima notes}}{#1}}}

\newcommand{\ret}{\mathsf{ret}}
\newcommand{\new}{\mathsf{new}}
\newcommand{\hnew}{h_\mathsf{new}}
\newcommand{\old}{\mathsf{old}}
\newcommand{\op}{\mathsf{op}}
\newcommand{\verifier}{\mathcal{V}}
\newcommand{\prover}{\mathcal{P}}
\newcommand{\key}{\mathsf{key}}
\newcommand{\nextkey}{\mathsf{nextKey}}
\newcommand{\node}{\mathsf{t}}
\newcommand{\parent}{\mathsf{p}}
\newcommand{\leaf}{\mathsf{f}}
\newcommand{\vl}{\mathsf{value}}
\newcommand{\balance}{\mathsf{balance}}
\newcommand{\lft}{\mathsf{left}}
\newcommand{\rgt}{\mathsf{right}}
\newcommand{\lbl}{\mathsf{label}}
\newcommand{\direction}{\mathsf{d}}
\newcommand{\oppositedirection}{\bar{\mathsf{d}}}
\newcommand{\found}{\mathsf{found}}
\newcommand{\mypar}[1]{\smallskip\noindent\textbf{#1.}\ \ \ }
\newcommand{\ignore}[1]{}

\newcommand{\langname}{ErgoScript\xspace}
\newcommand{\mixname}{ErgoMix\xspace}


\newcommand{\lst}[1]{\text{\lstinline[basicstyle={\ttfamily}]$#1$}}

\newcommand{\andnode}{\ensuremath{\mathsf{AND}}}
\newcommand{\ornode}{\ensuremath{\mathsf{OR}}}
\newcommand{\tnode}{\ensuremath{\mathsf{THRESHOLD}}}
\newcommand{\GF}{\ensuremath{\mathrm{GF}}}


\begin{document}


\title{Multi-Stage Contracts in the UTXO Model}


\author{Alexander Chepurnoy, Amitabh Saxena}
\institute{Ergo Platform \\\email{\{kushti,amitabh123\}@protonmail.ch}}

\maketitle


\begin{abstract}

Smart contract platforms such as Bitcoin and Ethereum allow writing programs that run on a decentralized computer. Bitcoin uses short-lived immutable data structures called UTXOs for data manipulation. Ethereum, on the other hand uses, long-lived mutable data structures called {\em accounts}. UTXOs are easier to handle, less error prone and scale better because the only operation we can do with them  is to create or destroy (i.e., spend) them. The code inside a UTXO is executed only once, when it is spent. Additionally, this code refers to only local context (i.e., it is stateless). In Ethereum's account based system, there is a shared global context which each account can access and modify, thereby causing side affects. 
However, the benefit of persistent storage offered by accounts makes up for these drawbacks. 
In this work, we describe how to emulate persistent storage in UTXO based systems using a technique called {\em transaction trees}. This allows us to emulate the functionality of account-based systems such as Ethereum without the overhead of accounts. 
We demonstrate this via several examples which include contracts for a Rock-Paper-Scissors game, crowdfunding and an initial coin offering (ICO). The contracts are created in a UTXO based smart contract platform called Ergo that supports transaction trees. 
\end{abstract}


\section{Introduction}

Smart contracts were envisioned in 1994 by Nick Szabo~\cite{szabo1997idea}, a legal scholar and cryptographer. He proposed the concept of self-executing contracts written in executable code and stored in a replicated manner on distributed computers that enforced the rules written in the code. 
Bitcoin~\cite{Nak08} can be seen as the first implementation of this concept using a fully decentralized ledger whose contracts primarily pertain to transfer and store of value, i.e., as a currency system. 
Ethereum~\cite{wood2014ethereum} is an example of a general-purpose smart contract platform. 

The limited application of Bitcoin allows optimizations focussed on long-term survivability and scalability. Firstly, all data and code is stored in short-lived immutable objects (called UTXOs~\cite{utxo}). A user can execute code inside a UTXO by supplying some input (which may contain additional code). A UTXO is destroyed once its code is executed (i.e., it is spent). Secondly, all computation is performed within a {\em local context}; any code pertaining to a UTXO can only operate on data for that UTXO and does not have access to the global state. Finally, the instruction set is limited enough to guarantee that the language is {\em Turing decidable}. In particular, every program of size $n$ terminates in $\Theta(n)$ steps. 

In contrast, Ethereum follows a different set of design principles in which the code and data is contained in long-lived mutable objects called {\em accounts}. This was done because UTXOs are stateless and do not provide persistent storage. 
Not only can Ethereum code modify data in its own account, but also trigger execution of code in other accounts. Thus, Ethereum code operates over a {\em shared global context} representing all existing accounts. 
Additionally, Ethereum's contract writing language is Turing complete with a cost component: every instruction incurs a non-refundable cost, called {\em gas}, that must be paid beforehand (any still executing contract is reset to its initial stage once the gas runs out). 
All these features allow sophisticated contracts. However, they also cause several scalability and security issues, which are still being addressed~\cite{Luu:2016:MSC:2976749.2978309,hackernoon17}. 


Firstly, we note that even the most advanced contracts written in Ethereum do not require Turing completeness~\cite{hackernoon16,hackernoon17}. Nor does being Turing complete imply suitability for writing complex contracts. For example, the Rule 110 \cite{Cook_2009,neary2006p} 
cellular automation is Turing complete but impractical for real-world use. 
Secondly, while UTXOs lack persistant storage, we can achieve the same by propagating data across multiple UTXOs using a technique called {\em transaction trees}. 
This was used to construct a Turing complete language by emulating Rule-110~\cite{CKM18a}.

While the results of~\cite{CKM18a} allow UTXO-based systems to emulate Ethereum-like functionality, at least in theory, a practical solution for the same is desirable. In particular, we need higher-level abstractions (instead of Rule-110) that enable UTXO-based systems to efficiently emulate Ethereum functionality and maintain Bitcoin's scalability. Ergo is one such platform.

\textbf{Context Enrichment.} The main idea behing Ergo is that of enriching the local context. In Bitcoin and other UTXO systems, the context is just the UTXO being processed. However, we can have various levels of local contexts.
In particular, for any UTXO based blockchain, we can define the following levels of context, each extending the previous:
\begin{enumerate}
	\item The current UTXO plus the blockchain height and time
	\item The current spending transaction (other inputs and outputs)
	\item The current block's solution.
	\item The current block (other sibling transactions)
\end{enumerate}

Any platform at Level 2 and above is suitable for transaction trees. In this regard, Bitcoin is at Level 1, while Ergo is at Level 3. 

Note that in Level 4 it is not possible to check validity of a transaction independently of other transactions. Hence it is more complex to implement. 

In this work we show via examples how to create efficient Ethereum-like contracts in the UTXO model using transaction trees in Ergo. 
The examples include a Rock-Paper-Scissors game, an Initial Coin Offering (ICO) campaign and a new primitive called {\em reversible addresses} for securely storing funds. The examples show how to create multi-stage contracts with evolving data and demonstrate how Ergo can cover a majority of use-cases where Ethereum is used.  


\section{Ergo Overview}
\label{ergo-overview}

Ergo uses Bitcoin's UTXO model but allows general-purpose smart contracts via a highly expressible (but still decidable) language called \langname. Ergo has better scalability than Bitcoin due to additional enahcements such as storage rent~\cite{chepurnoy2018systematic} and support for thin clients that can verify the entire blockchain with only a small amount of data. Unlike Ethereum and most other smart contract platforms which only allow imperative style programs, \langname supports functional programming and its syntax borrows heavily from Scala. 


We refer the reader to \langname resources for details~\cite{langrepo,tutorial,advtutorial}. Here we give only a brief overview.
Since Ergo follows the UTXO based model, all data and code is stored in immutable objects called {\em boxes}. As in Bitcoin, a transaction in Ergo can spend (destroy) multiple boxes and create new ones. 
A box is made of upto ten {\em registers} labelled $R_0, R_1,\ldots R_9$, four of which are mandatory. $R_0$ contains the monetary value, $R_1$ contains the {\em guard script}, $R_2$ contains assets (tokens) and $R_3$ contains a unique identifier of 34 bytes made up of a transaction ID and an output index. The guard script in $R_1$ encodes a spending condition, which must be satisfied for spending the box. Deploying a contract involves creating an unspent box with the relevant \langname code in $R_1$ and populating other registers if necessary. The contract is executed by spending the box. %, after which it gets destroyed. %Thus, a contract can be executed at most once.


Similar to Bitcoin, an \langname program also cannot access the global state and all computation must be done only using a local context. Unlike Bitcoin, this context is quite rich and allows access to the entire spending transaction~\cite{tutorial}. 
In particular, an \langname program defines the spending condition using predicates on the inputs and outputs of the transaction interleaved with Sigma protocols~\cite{Dam10}. Thus, an \langname program can enforce the spending transaction's structure (such as requiring that assets are transferrable only to a certain address). Additionally, the program can require the spender to prove knowledge of some secret via a Sigma protocol. \langname currently supports two such protocols. The first, called \texttt{proveDlog}, is a proof of knowledge of discrete log using the Schnorr identification scheme~\cite{Dam10}. The second, called \texttt{proveDHTuple}, is essentially a prove of discrete log equality~\cite{advtutorial}. All public keys (such as \texttt{alice} and \texttt{bob}) in the following sections are of type \texttt{proveDlog}. Similar to Bitcoin, a Pay-to-Script-Hash (P2SH) address in Ergo contains the hash of a script that must be provided when spending from that address. The script encodes the actual spending condition. Ergo also supports Pay-to-Script (P2S) address, where the actual script is encoded in the address. Since the size of a P2S address depends on the size of the script, all contracts in this paper use P2SH addresses.


 
 One useful feature of Ethereum is the ability to store and access a large amount of data, which Ergo also provides.
 
 However, Ergo contracts do not store the actual data in the blockchain. Rather, the data is stored offchain and a short digest is stored in the blockchain. A user wishing to access or modify this data must provide correct proofs of (non)-existence or modification for this digest, as in the ICO example of Section~\ref{ico}. 

\section{Transaction Trees}
A powerful feature of \langname is the ability to specify the spending transaction's structure in a fine-grained manner. 
Among the many things we can specify, the important ones are: (1) the number of input and output boxes, (2) the value of any box, and (3) the guarding script of any box. This allows us to create {\em transaction trees}, where the contract in an input box requires an output box to contain another predefined contract, which can then require its spending transaction's output to contain yet another predefined contract. We can use this technique to ensure that only some predefined sequence of contracts are possible. For example, the ``self-reproducing coins'' of~\cite{CKM18a} were created by requiring the output contract to be identical to the input one, thereby allowing the process to continue indefinitely. In a more realistic setting, we will use this to construct multi-stage protocols, where each stage corresponds to a contract that encodes what data and code must be carried over to the next stage. 

\textbf{Transaction Chains:} Before describing transaction trees, we describe a simpler primitive called transaction chains. A transaction chain is used for creating a multi-stage protocol whose code does not contain loops or `if' statements. A transaction chain is created as follows:
\begin{enumerate}
	\item Represent an Ethereum contract's execution using $n$ sequential steps, where each step represents a transaction that modifies its state.
The states before and after a transaction are the start and end nodes respectively of a directed graph, with the transaction as the edge joining them. 
	As an example, a 3-stage contract, such as the ICO example of Section~\ref{ico} is represented as:
	
	\begin{tikzpicture}
	
	\tikzset{vertex/.style = {shape=circle,draw,minimum size=1.5em}}
	\tikzset{edge/.style = {->,> = latex'}}
	% vertices
	\node[vertex] (a) at  (0,0) {$s_1$};
	\node[vertex] (b) at  (2,0) {$s_2$};
	\node[vertex] (c) at  (4,0) {$s_3$};
	%edges
	\draw[edge] (a) to (b);
	\draw[edge] (b) to (c);
	
	\end{tikzpicture}
	
	The states contain data and the code that was executed in the transaction.
	\item Hard-code state $n$'s code and data inside state $n-1$'s code. Then require the code of state $n-1$ to output a box containing state $n$'s code and data. An example is given in the following pseudocode:
	\begin{verbatim}
	out.propositionBytes == state_n_code &&  
	out.R4[Int].get == SELF.R4[Int].get // ensure data is propagated
	\end{verbatim}
	
	The above code uses the field \texttt{propositionBytes} of a box, which contains the binary representation of its guard script as collection of bytes.
	
	
	\item Repeat Step 2 by replacing $(n, n-1)$ by $(n-1, n-2)$ while $n > 2$.
\end{enumerate}

\textbf{Tranasction Trees:} A transaction tree is an extension of transaction chains where the code can contain `if' statements and {\em simple loops}, i.e., where some start and end nodes are the same. The following figure illustrates a transaction tree.

\begin{tikzpicture}

\tikzset{vertex/.style = {shape=circle,draw,minimum size=1.5em}}
\tikzset{edge/.style = {->,> = latex'}}
% vertices
\node[vertex] (a) at  (0,0) {$s_1$};
\node[vertex] (b) at  (2,0) {$s_2$};
\node[vertex] (c) at  (4,1) {$s_3$};
\node[vertex] (d) at  (4,-1) {$s_4$};
%edges
\draw[edge] (a) to (b);
\draw[edge] (b) to (c);
\draw[edge] (b) to (d);
\draw[black,thin,->] (a.80) arc (0:264:4mm);

\end{tikzpicture}

An `if' statement is handled using the following pseudocode. 

\begin{verbatim}
    if (condition) { out.propositionBytes == state_3_code }
    else { out.propositionBytes == state_4_code }
\end{verbatim}

A simple loop is a special case of the `if' statement:
\begin{verbatim}
    if (condition) { out.propositionBytes == state_2_code }
    else { out.propositionBytes == SELF.propositionBytes }
\end{verbatim}

\textbf{Transaction graphs:} Ergo supports a more advanced technique called {\em transaction graphs}, where cycles are allowed in contract references, as shown below. 

	\begin{tikzpicture}
	
	\tikzset{vertex/.style = {shape=circle,draw,minimum size=1.5em}}
	\tikzset{edge/.style = {->,> = latex'}}
	% vertices
	\node[vertex] (a) at  (0,-1) {$s_1$};
	\node[vertex] (b) at  (2,0) {$s_2$};
	\node[vertex] (c) at  (4,0) {$s_3$};
	\node[vertex] (d) at  (2,-2) {$s_4$};
	\node[vertex] (e) at  (4,-2) {$s_5$};
	%edges
	\draw[edge] (a) to (b);
	\draw[edge] (b) to (c);
	\draw[edge] (d) to (a);
	\draw[edge] (b) to (d);
	\draw[edge] (d) to (e);
	\draw[edge] (e) to (b);
\draw[black,thin,->] (a.80) arc (0:264:4mm);
		
	\end{tikzpicture}

Discussion of such contracts is beyond the scope of this work and we refer the reader to~\cite[Section 3.3.3]{advtutorial} for an example of such a contract. All the examples in this paper are based on transaction trees.

\section{Multi-Stage Contracts}

This section contains several examples of multi-stage contracts created using transaction trees.
Our first example is the classic Rock-Paper-Scissors game, which is often used to introduce and demonstrate Ethereum's capabilities.

\subsection{Rock-Paper-Scissors Game}
\label{rps}

Rock-Paper-Scissors is a 2-player game, where each player chooses a secret independently and the game is decided after the secrets are revealed. 
Let $a, b\in \mathbb{Z}_3$ be the secrets of Alice, Bob, with the understanding that (0, 1, 2) represent (rock, paper, scissors). If $a = b$ then the game is a draw, otherwise if $a-b \in \{1, -2\}$ then Alice wins else Bob wins. 

The first party to reveal the secret has a disadvantage, since the other party can adaptively choose and win. In the real world, both parties reveal their secrets simultaneously to prevent this. In the virtual world, however, this cannot be enforced. Hence this attack must be handled using {\em cryptographic commitments}, where the first party, Alice, does not initially reveal her secret, but rather only a commitment to that secret. 
The modified game using commitments is as follows:

\begin{enumerate}
	\item Alice commits to her secret $a$ by inputting her commitment $c = Comm(a)$.
	\item Bob inputs his public value $b$. At this stage, Alice knows if she won or lost.
	\item Alice opens her commitment and reveals $a$, after which the winner is decided.
\end{enumerate}

This works fine assuming that Alice is {\em well-behaved}, i.e., she always opens her commitment irrespective of whether she won or lost. In the real world, however, we also need to consider the possibility that Alice never opens her commitment. 
Border cases such as these make smart contracts quite tricky, because once deployed, it is not possible to add ``bug-fixes'' to them. In this example, we must penalize Alice (with a loss) if she does not open her commitment within some stipulated time.
The complete game is coded in \langname as follows.

The full game is made of two stages connected via contract chaining. In the first stage, Alice creates a {\em start-game} box that encodes her game rules. In the second stage, Bob spends the start-game box and creates two {\em end-game} boxes spendable by the winner. These new boxes indicate that the game has ended.

To start the game, Alice decides a game amount $x$ (of Ergo's primary token), which each player must contribute. She then selects a secret $s$ and computes a commitment $c=H(a||s)$ to $a$. Finally, she locks up $x$ tokens along with her commitment $c$ inside the start-game box protected by the following script:
\small{
	\begin{verbatim}
OUTPUTS.forall(
  {(out:Box) =>
    val b = out.R4[Byte].get
    val bobDeadline = out.R6[Int].get
    bobDeadline >= HEIGHT+30 && out.value >= SELF.value &&
    (b == 0 || b == 1 || b == 2) && 
    out.propositionBytes == outScript
  }
) && OUTPUTS.size == 2 && OUTPUTS(0).R7[SigmaProp].get == alice &&
OUTPUTS(0).R4[Byte].get == OUTPUTS(1).R4[Byte].get // same b
	\end{verbatim}
}


The above code requires that the spending transaction must create exactly two outputs, one paying to each player in the event of a draw or both paying to the winner otherwise. In particular, the code requires that (1) register $R_7$ of the first output must contain Alice's public key (for use in the draw scenario), (2) register $R_4$ of each output must contain Bob's choice, and (3) each output must contain at least $x$ tokens protected by \texttt{outScript}, which is given below:

\small{
\begin{verbatim}
val s = getVar[Coll[Byte]](0).get  // Alice's secret byte string s
val a = getVar[Byte](1).get  // Alice's secret choice a
val b = SELF.R4[Byte].get    // Bob's public choice b
val bob = SELF.R5[SigmaProp].get // Bob's public key
val bobDeadline = SELF.R6[Int].get // after this, Bob wins by default
val drawPubKey = SELF.R7[SigmaProp].get
val valid_a = (a == 0 || a == 1 || a == 2) 
val validCommitment = blake2b256(s ++ Coll(a)) == c   
val validAliceChoice = valid_a && validAliceChoice
val aliceWins = (a - b) == 1 || (a - b) == -2
val receiver = if (a == b) drawPubKey else (if (aliceWins) alice else bob)
(bob && HEIGHT > bobDeadline) || (receiver && validAliceChoice)
\end{verbatim}
}


The above code protects the two end-game boxes that Bob generates. The condition \texttt{(bob \&\& HEIGHT > bobDeadline)} guarantees that if Alice does not open her commitment before a certain deadline, then Bob automatically wins. For simplicity, we ignore fee and refer to Sections~\ref{reversible} and~\ref{ico} for code that handles fee. 
Note that Bob has to ensure that $R_7$ of the second output contains his public key. Additionally, he must ensure that $R_5$ of both outputs contains his public key (see below). We don't encode these conditions because if Bob doesn't follow the protocol, he will automatically lose.

\subsection{Reversible Addresses}
\label{reversible}

Our next example describes {\em reversible addresses} that have anti-theft features in the following sense: any funds sent to a reversible address can only be spent in way that allows payments to be reversed for a certain time. The idea was originally proposed for Bitcoin and requires a hard fork~\cite{raddress}. In Ergo, however, this can be done natively.

To motivate this feature, imagine managing the hot-wallet (for instance, of an exchange) used for handling withdraws of customers. A hot-wallet is an address for which the private key is stored on the server. Such addresses are necessary for facilitating automated withdrawals. Being a hot-wallet, its private key is susceptible to compromise and funds being stolen. 
We want to ensure that we are able to recover any such stolen funds, provided that the breach is discovered within a specified time (such as 24 hours) of the first unauthorized withdraw. This can be achieved using reversible addresses.

Assume that \texttt{alice} is the public key of the hot-wallet and \texttt{carol} is the public key of the trusted party. The private key of \texttt{carol} will be needed for reversing payments and must be stored offline. Let \texttt{b} be the estimated number of blocks in a 24 hour period. 
Let Bob with public key \texttt{bob} be a customer wishing to withdraw funds, which will be paid out by the hot-wallet. 

In Ethereum, we can do this by sending funds to an account having with a contract $C_{b}$ that allows \texttt{carol} to withdraw funds at least \texttt{b} blocks and after that they can only be withdrawn by \texttt{bob}. We could use the same account (contract instance) for multiple withdraws by Bob, but the optimal way is to have a new account for each withdraw, emulating the UTXO model. The funds for this must also come from another account with a contract $C_{a}$ that ensure that withdraw can only be done to a contract with the structure of $C_{b}$. 
In Ergo, this is done by a two-stage protocol, where the second stage implements $C_{b}$ and the first stage implements $C_{a}$. 

The following script called \texttt{withdrawScript} implements the second stage. This will be the guarding script of the hot-wallet's withdraw transaction paying to \texttt{bob}.

\small{
\begin{verbatim}
val bob = SELF.R4[SigmaProp].get    // public key of customer withdrawing
val bobDeadline = SELF.R5[Int].get  // max locking height
(bob && HEIGHT > bobDeadline) || (carol && HEIGHT <= bobDeadline)
\end{verbatim}
}

This above script is referenced in the first stage script given next. Let \texttt{feeScript} be the script paying fee and \texttt{maxFee} be the maximum allowed fee. 

\small{
\begin{verbatim}
val isChange = {(b:Box) => b.propositionBytes == SELF.propositionBytes}
val isWithdraw = {(b:Box) => 
  b.R5[Int].get >= HEIGHT + blocksIn24h && 
  b.propositionBytes) == withdrawScript
}
val isFee = {(b:Box) => b.propositionBytes) == feeScript}
val isValid = {(b:Box) => isChange(b) || isWithdraw(b) || isFee(b)}
val totalFee = OUTPUTS.fold(0L, 
  {(x:Long, b:Box) => if (isFee(b)) x + b.value else x}
)
alice && OUTPUTS.forall(isValid) && totalFee <= maxFee
\end{verbatim}
}

The reversible address is the P2SH address of the above script. Any funds sent to this address are subject to the withdraw rules that we desire.
%The sender (\texttt{alice}) must ensure that register $R_4$ of the new box contains \texttt{bob}. 

In the normal case, Bob will spend the box after roughly \texttt{blocksIn24h} blocks. If an unauthorized transaction from the hot-wallet is detected, an abort procedure is triggered using the private key of \texttt{carol} and funds in any unspent boxes sent from the hot-wallet are diverted to a secure address. %Additionally, boxes currently controlled by \texttt{alice} also need to be sent secure addresses. 
Note that the trusted party (\texttt{carol}) is bound to the hot-wallet address. A new address is needed for a different trusted party. 

Although such addresses are designed for securing hot-wallet funds, the may have other applications. One example is for automated-release escrow payments in online shopping, where \texttt{carol} can be the public key of any mutually agreed adjudicating party. 

\subsection{Cold-Wallet Addresses}

\label{cold-wallet}

A hot-wallet such as the above is usually funded from another {\em cold-wallet}, which is an address whose private key is store offline. While reversible addresses are enough if the theft is quickly detected, there is always a chance that detection is delayed. A second line of defence is to ensure that only a small fraction of funds is present in the hot-wallet, with the remainder in the cold-wallet. This ensures that even if a theft from the hot-wallet is successful, a majority of the funds remains safe. We can do this by using a special cold-wallet address that encodes an advanced form of Bitcoin's multi-signature contract. 
The idea is to restrict funding based on how many keys are used for signing. As an example, assume that the address is protected by 2 private keys, corresponding to the public keys \texttt{alice} and \texttt{bob} such that the following conditions must hold:

\begin{enumerate}
	\item One key can spend at most 1\% or 100 ergs (whichever is higher) in one day.
	\item If both keys are spending then there are no restrictions. 
\end{enumerate}

An address restricting funds in this manner is called a {\em cold-wallet address}. 
This is our first example of a {\em stateful contract}, where we need access to 
data of prior transactions, not just the current one. While creating such contracts in Ethereum is easy because of persistent storage, doing the same in Ergo is non-trivial.
The trick is to accumulate data from previous transactions in the registers ($R_4, R_5$ in our example).

Let \texttt{blocksIn24h} be the number of blocks in 24 hours. Instead of hardwiring 1\% and 100 ergs, we will use the named constants \texttt{percent} and \texttt{minSpend} respectively. The cold-wallet address is the P2SH address of the following script:
\begin{verbatim}
val storedStartHeight = SELF.R4[Int].get // when the period started
val creationHeight = SELF.creationInfo._1 // creation height
val startHeight = min(creationHeight, storedStartHeight) 
val notExpired = HEIGHT - startHeight <= blocksIn24h // period valid
val min = SELF.R5[Long].get // min Balance needed in this period

val ours:Long = SELF.value - SELF.value * percent / 100
val keep = if (ours > minSpend) ours else 0L // handling topups
val nStart:Int = if (notExpired) start else HEIGHT
val nMin:Long = if (notExpired) min else keep

val out = OUTPUTS(0)  
val valid = INPUTS.size == 1 && 
            out.propositionBytes == SELF.propositionBytes &&
            out.value >= nMin && out.R4[Int].get >= nStart && 
            out.R5[Long].get == nMin
 
(alice && bob) || ((alice || bob) && min >= keep && (nMin == 0 || valid))
\end{verbatim}

Spending from this address is done in periods of 24 hours or more such that the maximum spendable is a fixed fraction of the amount at the beginning of the period. We require the spending transaction to have an output with value greater than the minimum (stored in $R_5$) and paying back to the same address. The start of the current period is stored in $R_4$. Both registers are copied to the new output within the same period and get new values if the current period has expired. 

The above two-party contract can be extended for multiple parties, such as one between three parties and enforcing the following rules:

\begin{enumerate}
	\item One key can spend at most 1\% or 100 ergs (whichever is higher) in one day.
	\item For two keys the amount is 10\% or 1000 ergs (whichever is higher).
	\item If all three keys are spending then there are no restrictions. 
\end{enumerate}

We refer the reader to the \langname repository~\cite{langrepo} for the complete code.

%We leave this contract as an exercise to the reader.
\subsection{Revenue Sharing Addresses}
\label{revenue-sharing}

Another type of contract often used in Ethereum is for {\em revenue sharing}, which is used to
enforce that all incoming funds are shared in a pre-defined manner.
Assume that Alice, Bob and Carol agree to share revenue in the ratios of 50\%, 30\% and 20\% respectively. Let \texttt{alice}, \texttt{bob} and \texttt{carol} be their public keys. The following describes how to create an address that automatically enforces this in Ergo. First define a collection: 
\begin{verbatim}
val spenders = Coll((alice, 50), (bob, 30), (carol, 20))
\end{verbatim}

Let \texttt{feePropBytes} to be the script that pays transaction fees and let \texttt{fee} be the specified fee. 
% There may be advantages of storing the public keys in a register rather than hardwiring them to address
The revenue sharing address is the Pay-to-Script-Hash (P2SH) address of the following script.

\begin{verbatim}
val feeBox = OUTPUTS(0)        // assume that first output pays fee
val validFeeBox = feeBox.propositionBytes == feePropBytes
val amt = SELF.value - fee     // balance remaining after deducting fee
val ratios = spenders.map({(e:(Coll[Byte], Int)) => e._2})
val total = ratios.fold(0, {(l:Int, r:Int) => l + r})
val validOuts = spenders.map({(e:(Coll[Byte], Int)) =>
   val pubKey = e._1        // public key of receiver
   val ratio = e._2         // ratio of funds to be paid to public key
   val share = amt / total * total   // actual amount based on ratio
   OUTPUTS.exists(
     {(b:Box) => b.value >= share && b.propositionBytes == pubKey}
   )
})
validOuts.fold(true, {(l:Boolean, r:Boolean) => l && r}) && validFeeBox
\end{verbatim}

The script ensures that any funds sent to this address can only be spent in a transaction that creates boxes paying to Alice, Bob and Carol in the ratios defined above (after deducting fee).
We can extend this script to define a more advanced script for {\em salary payments}, where the incoming funds are divided on a weekly or monthly basis capped to a maximum. 

\subsection{Crowdfunding}
\label{crowfunding}

We consider the simplest crowdfunding scenario. In this example, a crowdfunding project with a known public key is considered successful if it can collect unspent outputs with a total value not less than a certain amount before a certain height. A project backer creates an output box protected by the following statement: {\em the box can be spent if the spending transaction has the first output box protected by the project’s key and amount no less than the target amount.} Then the project can collect (in a single transaction) the biggest
backer output boxes with a total value not less than the target amount (it
is possible to collect up to 22,000 outputs in Ergo, which is enough for a big campaign). For the remaining outputs, it can construct
follow-up transactions.

More concretely, a project backer (with key \texttt{backerPubKey}) wishes to give money to the project (with key \texttt{projectPubKey}), but only if the project raises enough money (at least \texttt{minToRaise}) from other sources by a deadline \texttt{maxDeadline} (expressed in terms of \texttt{HEIGHT}). The backer creates an output box protected by the following script which encodes two conditions: one for the case the deadline has passed (enabling the
 backer to get the money back) and one for the case it succeeded (enabling the project to spend
 the money if the amount is at least \texttt{minToRaise} before the deadline). In order to ensure enough  money has been raised, the script will search the output collection for a box with a sufficient value going to \texttt{projectPubKey}. To verify that the funds are indeed going to the project, the code will compare the guard script of the output box with \texttt{proveDlog(projectPubKey)}; this script can be obtained by \texttt{projectPubKey.propBytes}. The following is the script protecting the backer funds.
\begin{verbatim}
    val fundraisingFailure = HEIGHT >= deadline && backerPubKey
    val enoughRaised = {(outBox: Box) =>
        outBox.value >= minToRaise &&
        outBox.propositionBytes == projectPubKey.propBytes
    }
    val fundraisingSuccess = HEIGHT < deadline && projectPubKey &&
    OUTPUTS.exists(enoughRaised)
    fundraisingFailure || fundraisingSuccess
\end{verbatim}

The values of \texttt{deadline}, \texttt{minToRaise}, and the two public keys are hardwired into the script at compile time.
 
\subsection{Initial Coin Offering}
\label{ico}

Another popular use-case of Ethereum is an Initial Coin Offering (ICO) contract. %, possibly implementing the ERC-20 standard~\cite{erc20} 
An ICO mirrors an Initial Public Offering (IPO) and provides a mechanism for a project to collect funding in some tokens and then issue ``shares'' (in the form of some other tokens) to investors. Generally, an ICO comprises of 3 stages:
\begin{enumerate}
	\item {\em Funding:} During this period, investors are allowed to fund the project.
	\item {\em Issuance:} A new asset token is created and issued to investors.
	\item {\em Withdrawal:} Investors can withdraw their newly issued tokens. 
\end{enumerate}

Compared to the previous examples, our ICO contract is quite complex, since it involves multiple stages and parties. 
The number of investors may run into thousands, and the naive solution would store this data in the contract, as in the ERC-20 standard~\cite{erc20}. 
Unlike Ethereum, Ergo does not permit storing large datasets in a contract. Rather, we store only a 40-bytes header of (a key, value) dictionary, that is authenticated like a Merkle tree~\cite{RMCI17}. To access some elements in the dictionary, or to modify it, a spending transaction should provide lookup or modification proofs. This allows a contract to authenticate large datasets using very little storage and memory. 

\textbf{Funding:} 
The project initiates the ICO by creating a box with the guard script given in Appendix~\ref{ico-funding}. The box also contains a authenticating value for an empty dictionary of (investor, balance) pairs in $R_5$, where investor is the hash of a script that will guard the box with the withdrawn tokens (once the funding period ends). 

The first funding transaction spends this box and creates a box with the same script and updated data. Further funding transactions spend the box created from the previous funding transaction. The box checks that it is first input of each funding transaction, which must have other inputs belonging to investors. The investor inputs contain a hash of the withdraw script in register $R_4$. 

The script also checks (via proofs) that hashes and monetary values of the investing inputs are correctly added to the dictionary of the new box, which must be the first of two outputs with the correct amount of ergs (the second output pays a predefined fee). 
In this stage, which lasts at least till height 2,000, withdraws are not permitted and ergs can only be put into the project. 
The first transaction with height of 2,000 or more should keep the same data but change the output's script as described next.

\textbf{Issuance:}
This stage has only one spending transaction to get to the next stage (the withdrawal stage). The spending transactions makes the following modifications. Firstly, it changes the list of allowed operations on the dictionary from ``inserts only'' to ``removals only'', as the next stage only deals with removing entries from the dictionary.

Secondly, the contract checks that the proper amount of ICO tokens are issued. In Ergo, it is allowed to issue one new kind of token per transaction, and the identifier of the token should be equal to the (unique) identifier of the first input box. The issuance sub-contract checks that a new token has been issued, and the amount of it is equal to the amount of nano-ergs collected by the ICO at till now. Thirdly, the contract checks that a spending transaction is indeed re-creating the box with the guard script corresponding to the next stage, the withdrawal stage.

Finally, the project should withdraw collected ergs, and of course, each spending transaction should pay a fee. Thus, the sub-contract checks that the spending transaction has indeed 3 outputs (one each for the project tokens box, the ergs withdrawal box, and the fee box), and that the first output and output is carrying the tokens issued. As we do not specify project money withdrawal details, we require a project signature on the spending transaction. The complete script is given in Appendix~\ref{ico-issuance}.

\textbf{Withdrawal:}
Investors are now allowed to withdraw ICO tokens under a guard script whose hash is stored in the dictionary. Withdraws are done in batches of $N$. A withdrawing transaction, thus, has $N + 2$ outputs; the first output carries over the withdrawal sub-contract and balance tokens, the last output pays the fee and the remaining $N$ outputs have guard scripts and token values as per the dictionary. The contract, given in Appendix~\ref{ico-withdrawal}, requires two proofs for the dictionary elements: one proving that values to be withdrawn are indeed in the dictionary, and the second proving that the resulting dictionary does not have the withdrawn values. 


\textbf{Fine-tuning:} Please note that there are many nuances our example contract ignores. For example, anyone listening to the blockchain is allowed to create spending transactions during the funding and withdrawal stages. In the real-world, additional signature from the project or a trusted arbiter might be used. Also, there is no self-destruct after the withdrawal contract, so the box will live until being destroyed via the in-built storage rent mechanism of Ergo~\cite{tutorial}, potentially for decades or even centuries. 
%For the funding stage, it would be reasonable to have an additional input from the project to cover the fee.

\section{Conclusion}

We gave examples demonstrating that, despite being UTXO-based, Ergo can support complex multi-stage contracts found in Ethereum. In particular, we described  contracts for the following examples: 

\begin{enumerate}
	\item A Rock-Papers-Scissors game with provable fairness (Section~\ref{rps}).
	\item Reversible Addresses having anti-theft features (Section~\ref{reversible}).
	\item Cold-wallet addresses that restrict how much can be spent (Section~\ref{cold-wallet}).
	\item Revenue-sharing and salary contracts that enforce a predefined distribution of any funds sent to them (Section~\ref{revenue-sharing}).
	\item Crowdfunding contracts that allow backers to securely fund a project (Section~\ref{crowfunding})
	\item A full featured ICO that accepts funding in ergs (Section~\ref{ico}). 

\end{enumerate}

The last three examples (revenue sharing, crowdfunding and ICO) can be combined along with a voting protocol to create a fully functioning {\em Decentralized Autonomous Organization} (DAO).


The examples used the idea of {\em transaction trees} to emulate persistent storage by linking several UTXOs containing small pieces of code to form a large multi-stage protocol. 
We refer the reader to \langname repository and tutorials~\cite{langrepo,tutorial,advtutorial} for additional examples of multi-stage contracts, including a Local Exchange Trading Systems (LETS), non-interactive mixing, atomic swaps and many more.

\bibliographystyle{unsrt}
\bibliography{main}

\appendix

\section{The ICO Campaign}

\subsection{The Funding Script}
\label{ico-funding}

The following is the script protecting the ICO funding box.

\small{\begin{verbatim}
val selfIndexIsZero = INPUTS(0).id == SELF.id
val proof = getVar[Coll[Byte]](1).get
val inputsCount = INPUTS.size
val toAdd = INPUTS.slice(1, inputsCount).map(
  {(b: Box) =>
    val pk = b.R4[Coll[Byte]].get
    val value = longToByteArray(b.value)
    (pk, value)
  }
)
val modifiedTree = SELF.R5[AvlTree].get.insert(toAdd, proof).get
val expectedTree = OUTPUTS(0).R5[AvlTree].get
val properTreeModification = modifiedTree == expectedTree
val outputsCount = OUTPUTS.size == 2
val selfOutputCorrect = 
  if (HEIGHT < 2000) OUTPUTS(0).propositionBytes == SELF.propositionBytes
  else OUTPUTS(0).propositionBytes == issuanceScript 
val feeOutputCorrect = 
  OUTPUTS(1).value <= 1 && OUTPUTS(1).propositionBytes == feeBytes
val outputsCorrect = outputsCount && feeOutputCorrect && selfOutputCorrect

selfIndexIsZero && outputsCorrect && properTreeModification
\end{verbatim}
}

The \texttt{issuanceScript} variable contains code of the issuance stage script.

\subsection{The Issuance Script}
\label{ico-issuance}
The following is the code of \texttt{issuanceScript}.
\small{
\begin{verbatim}
val openTree = SELF.R5[AvlTree].get
val closedTree = OUTPUTS(0).R5[AvlTree].get
val digestPreserved = openTree.digest == closedTree.digest
val keyLengthPreserved = openTree.keyLength == closedTree.keyLength
val valueLengthPreserved = 
              openTree.valueLengthOpt == closedTree.valueLengthOpt
val treeIsClosed = closedTree.enabledOperations == 4
val tokenId: Coll[Byte] = INPUTS(0).id
val tokensIssued = OUTPUTS(0).tokens(0)._2
val outputsCountCorrect = OUTPUTS.size == 3
val secondOutputNoTokens = OUTPUTS(0).tokens.size == 1 && 
                           OUTPUTS(1).tokens.size == 0 && 
                           OUTPUTS(2).tokens.size == 0
val correctTokensIssued = SELF.value == tokensIssued
val correctTokenId = OUTPUTS(0).R4[Coll[Byte]].get == tokenId &&
                     OUTPUTS(0).tokens(0)._1 == tokenId
val valuePreserved = outputsCountCorrect && secondOutputNoTokens && 
                     correctTokensIssued && correctTokenId
val stateChanged = OUTPUTS(0).propositionBytes == withdrawScript
val treeIsCorrect = digestPreserved && valueLengthPreserved && 
                    keyLengthPreserved && treeIsClosed

projectPubKey && treeIsCorrect && valuePreserved && stateChanged
\end{verbatim}
}

The variable \texttt{withdrawScript} contains the code of the withdrawl stage script.
\subsection{The Withdrawal Script}
\label{ico-withdrawal}

The following is the code of \texttt{withdrawScript}:
\small{
\begin{verbatim}
val removeProof = getVar[Coll[Byte]](2).get
val lookupProof = getVar[Coll[Byte]](3).get
val withdrawIndexes = getVar[Coll[Int]](4).get
val out0 = OUTPUTS(0)
val tokenId: Coll[Byte] = SELF.R4[Coll[Byte]].get
val withdrawals = withdrawIndexes.map(
  {(idx: Int) =>
    val b = OUTPUTS(idx)
    if (b.tokens(0)._1 == tokenId) 
      (blake2b256(b.propositionBytes), b.tokens(0)._2)
    else 
      (blake2b256(b.propositionBytes), 0L)
  }
)
val withdrawValues = withdrawals.map(
  {(t: (Coll[Byte], Long)) => t._2}
)
val withdrawTotal = withdrawValues.fold(0L, 
  {(l1: Long, l2: Long) => l1 + l2 }
)
val toRemove = withdrawals.map({(t: (Coll[Byte], Long)) => t._1})
val initialTree = SELF.R5[AvlTree].get
val removedValues = initialTree.getMany(toRemove, lookupProof).map(
  {(o: Option[Coll[Byte]]) => byteArrayToLong(o.get)}
)
val valuesCorrect = removedValues == withdrawValues
val modifiedTree = initialTree.remove(toRemove, removeProof).get
val expectedTree = out0.R5[AvlTree].get
val selfTokensCorrect = SELF.tokens(0)._1 == tokenId
val selfOutTokensAmount = SELF.tokens(0)._2
val soutTokensCorrect = out0.tokens(0)._1 == tokenId
val soutTokensAmount = out0.tokens(0)._2
val tokensPreserved = selfTokensCorrect && soutTokensCorrect && 
                      soutTokensAmount + withdrawTotal == selfOutTokensAmount
val properTreeModification = modifiedTree == expectedTree
val selfOutputCorrect = out0.propositionBytes == SELF.propositionBytes

properTreeModification && valuesCorrect && selfOutputCorrect && tokensPreserved
\end{verbatim}
}
%\subsection{Possible Enhancements}

\end{document}
